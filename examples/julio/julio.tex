%% -*- coding: utf-8; -*-

%% \listfiles

\documentclass[
  doutorado,
  brazilian,
  numeric
]{ThesisPUC}

  %% \usepackage[brazilian]{babel}      %% in ThesisPUC.cls
  %% \usepackage[utf8]{inputenc}        %% .
  %% \usepackage[T1]{fontenc}           %% .
  %% \usepackage{lmodern}               %% .
  %% \usepackage[pdftex]{graphicx}	%% .

  \usepackage{tabularx}
  \usepackage{multirow}
  \usepackage{multicol}
  \usepackage{colortbl}
  \usepackage[%
    dvipsnames,
    svgnames,
    x11names,
    fixpdftex
  ]{xcolor}
  \usepackage{numprint}
  \usepackage{textcomp}
  \usepackage{booktabs}
  \usepackage{amsmath}
  \usepackage{enumitem}
  \usepackage{amssymb}
  \usepackage{textcomp}

%%% Other commands

\newcommand{\Rset}{\mathbb{R}}
\newcommand{\Zset}{\mathbb{Z}}

\npthousandsep{}
\npdecimalsign{,}

\modotabelas{figtab} %% [nada, fig, tab ou figtab]
%\modoabreviacoes{none} %% [none ou use] %% Default is [use]

\setcounter{tocdepth}{3}
\setcounter{secnumdepth}{3}


%%%
%%% My files
%%%

% -*- coding: iso-8859-1; -*-

%%%
%%% Newcommands
%%%

\newcommand{\degree}{\ensuremath{^\circ}}

\newcommand{\cetem}{Centro de Tecnologia Mineral}

\newcommand{\mybulletOB}{%
  % \textbullet
  % \checkmark
  $\triangleright$
  %\textopenbullet
}

\newcolumntype{L}{>{\raggedright \arraybackslash}X}
\newcolumntype{R}{>{\raggedleft \arraybackslash}X}
\newcolumntype{C}{>{\centering \arraybackslash}X}
\newcolumntype{M}[1]{>{\centering\hspace{0pt}}m{#1}}

\newcommand{\mrcel}[2]{%
\begin{tabular}[c]{@{}c@{}}#1\\#2\end{tabular}}

\newcommand{\mrcell}[2]{%
\begin{tabular}[l]{@{}l@{}}#1\\#2\end{tabular}}

\newcommand{\mrcelthree}[3]{%
\begin{tabular}[c]{@{}c@{}c@{}}#1\\#2\\#3\end{tabular}}

\newcommand{\mrcelcolorg}[2]{%
\begin{tabular}{l}\rowcolor{Gainsboro}#1\\#2\end{tabular}}

\newcommand{\mytbcimg}[3]{%
  \multicolumn{1}{C}{\parbox[c]{#1}{\includegraphics[width=#2]{#3}}}}


%%%
%%% Titulos
%%%

\tituloglossario{Lista de Abreviaturas}

\autor{Julio César Álvarez Iglesias}
\autorR{Álvarez Iglesias, Julio César}

\orientador{Sidnei Paciornik}{Prof.}
\orientadorR{Paciornik, Sidnei}
% Se o departamento do orientador for diferente do departamento do autor, descomente a linha seguinte e coloque o nome e sigla da instituicao do orientador.
%\orientadorInst{nome da instituicao}{sigla}

\coorientador{Otávio da Fonseca Martins Gomes}{Dr.}
\coorientadorR{da Fonseca Martins Gomes, Otávio}
\coorientadorInst{Centro de Tecnologia Mineral}{CETEM/MCTI}

\titulo{Desenvolvimento de um sistema de microscopia digital para
  classificação automática de tipos de hematita em minério de ferro}

\titulouk{Development of a digital microscopy system for automatic
  classification of hematite types in iron ore}

%% \subtitulo{Aqui vai o subtitulo caso precise}

\dia{9}
\mes{Agosto}
\ano{2012}

\cidade{Rio de Janeiro}
\CDD{620.11}
\departamento{Engenharia Química e de Materiais}
\programa{Engenharia de Materiais e de Processos Químicos e Metalúrgicos}
\centro{Centro Técnico Científico}
\universidade{Pontifícia Universidade Católica do Rio de Janeiro}
\uni{PUC-Rio}

%%%
%%% Banca
%%%

\banca{%
  \membrodabanca{Paulo Roberto Gomes Brandão}{Prof.}
    {Universidade Federal de Minas Gerais}{UFMG}
  \membrodabanca{Leonardo Evangelista Lagoeiro}{Prof.}
    {Universidade Federal de Ouro Preto}{UFOP}
  \membrodabanca{Reiner Neumann}{Dr.}
    {Centro de Tecnologia Mineral}{CETEM/MCTI}
  \membrodabanca{Marcos Henrique de Pinho Maurício}{Dr.}
    {Departamento de Engenharia Química e de Materiais}{PUC-Rio}
}

%%%
%%% Currículo
%%%

\curriculo{%
  Graduou-se em Física pela Universidade da Havana (Havana,
  Cuba). Fez mestrado no Departamento de Engenharia de Materiais da
  PUC-Rio, especializando-se na área de microscopia digital,
  processamento e análise digital de imagens. Desenvolveu um grande
  número de rotinas de processamento digital de imagens para a
  companhia Vale na identificação e caracterização de minério de ferro
  assim como seus aglomerados.}

%%%
%%% Agradecimentos (LEMBRETE AOS ALUNOS BOLSISTAS. Não esqueçam de agradecer às agências ou órgãos de fomento à pesquisa que apoiaram a realização do seu trabalho.)
%%%

\agradecimentos{%
  Em primeiro lugar, gostaria de agradecer especialmente a meu
  orientador Prof. Sidnei Paciornik e coorientador Dr. Otávio
  Gomes. Em particular, quero manifestar o meu apreço ao Sidnei que,
  além de orientador, tem sido um amigo. Eu não tenho palavras para
  expressar minha sincera gratidão a ele, mas eu me considero
  afortunado por ter trabalhado com ele nos últimos seis anos; dois
  deles durante o meu mestrado. Agradecimentos especiais ao meu
  coorientador Otávio Gomes por ter me guiado durante esta caminhada
  com seu extenso conhecimento e entusiasmo.

  Eu tenho a sorte de ter a oportunidade de colaborar com colegas de
  laboratório muito competentes que sem dúvida têm contribuído
  significativamente para essa tesse. Eles também têm sido os
  responsáveis por criar um ambiente agradável e estimulante para
  trabalhar durante esses quatro anos de minha
  pesquisa. Individualmente, gostaria de agradecer a Karen Augusto por
  ter sacrificado longas horas na coleta de dados e análise de vários
  resultados importantes para a minha tese, a David Pirotte pelo
  suporte com as questões técnicas relacionadas ao uso do Latex, a
  Debora Turon pelos momentos compartilhados e aprendizado dividido,
  assim como a Luciana Carneiro.

  Eu também devo muito a Maria Beatriz Vieira, nossa colaborada da
  Vale, que gerenciou e nos permitiu o acesso a este desafiante
  problema. Certamente contar com a sua colaboração foi decisiva para
  o impacto e contribuição da minha tese no contexto da indústria
  brasileira.

  Agradeço ao CETEM/MCTI pela parceria e pelo apoio na aquisição e
  processamento dos resultados usados nesta tese.

  Também sou grato aos membros da minha banca examinadora, Prof. Paulo
  Roberto Gomes Brandão, Prof. Leonardo Evangelista Lagoeiro,
  Dr. Reiner Neumann e Dr. Marcos Henrique de Pinho Maurício, que
  comprometeram muito generosamente seu tempo e conhecimento para
  avaliar a minha tese.

  Eu também gostaria de agradecer à minha esposa, por cada minuto de
  amor, estímulo, paciência, encorajamento e força. A meus pais pelo
  apoio, mesmo a muitos quilômetros de distância, que tem sido
  essencial para encontrar forças e continuar lutando pelos meus
  objetivos. Palavras não podem expressar a imensidão da gratidão que
  tenho por eles. À minha família em geral pela força e confiança.

  Por último, mas não menos importante, gostaria de agradecer à CAPES
  pelo apoio financeiro e à PUC-Rio pela bolsa de isenção de
  mensalidades do doutorado.}

%%%
%%% Prechaves
%%%

\catalogprekeywords{%
  \catalogprekey{Engenharia Química}%
  \catalogprekey{Engenharia de Materiais}%
}

%%%
%%% Chaves
%%%

\chaves{%
  \chave{Minério de Ferro}
  \chave{Cristais de Hematita}
  \chave{Microscopia Digital}
  \chave{Análise de Imagens}
  \chave{Classificação}
  \chave{Microscopia de Luz Polarizada}
}

\chavesuk{%
  \chave{Iron Ore}%
  \chave{Hematite Crystals}%
  \chave{Digital Microscopy}%
  \chave{Image Analysis}%
  \chave{Classification}%
  \chave{Polarized Light Microscopy}%
}

%%%
%%% Resumo
%%%

\resumo{%
  O minério de ferro é um material policristalino oriundo de processos
  naturais complexos durante tempos geológicos, que dão origem a
  características intrínsecas e comportamento industrial variado.  A
  grande maioria dos minérios de ferro brasileiros é essencialmente
  hematítico. A hematita pode ser classificada como lobular, lamelar,
  granular, microcristalina ou martita. Na indústria mineral, esta
  caracterização é tradicionalmente realizada por operadores humanos a
  partir da observação de amostras no microscópio ótico, sujeita a
  grandes variações. Assim, é relevante desenvolver um procedimento
  que permita a discriminação dos diferentes tipos de hematita e a
  medida de características tais como o tamanho de cristal. Esta tese
  propõe um sistema que mede e classifica automaticamente tipos
  texturais de hematita baseado no processamento e na análise de
  imagens de microscopia ótica, em campo claro, polarização linear e
  polarização circular. Foram desenvolvidas rotinas para aquisição,
  registro, segmentação, reconhecimento e análise morfológica de
  cristais de hematita. A segmentação automática de cristais de
  hematita foi baseada no cálculo da distância espectral, a fim de
  controlar o crescimento de regiões partindo das sementes. Os
  resultados da identificação dos cristais obtidos, tanto nas imagens
  obtidas com polarização linear quanto com polarização circular,
  foram muito promissores. Atributos de tamanho e forma dos cristais
  identificados foram obtidos. Estes dados foram usados como conjunto
  de treinamento para classificadores supervisionados, permitindo
  reconhecer as classes de hematita granular, lamelar e lobular. Taxas
  de acerto globais próximas a 98\% foram obtidas, tanto para
  autovalidação, quanto para a validação cruzada.}

\resumouk{%
  Iron ore is a polycrystalline material created by complex natural
  processes during geological periods, which give rise to intrinsic
  characteristics and varied industrial behavior. The vast majority of
  the Brazilian iron ores belong essentially to the hematitic
  type. Hematite can be classified as lobular, lamelar, granular,
  micro-crystalline or martite. In the mineral industry, the
  characterization of iron ore and its agglomerates is traditionally
  developed by human operators from the observation of samples under
  the optical microscope, which may suffer large variations. Thus, it
  is important to develop a procedure that allows the discrimination
  of the different hematite types and the measurement of
  characteristics such as crystal size. The present thesis proposes a
  system for the automatic classification of hematite textural types,
  based on digital processing and analysis of optical microscopy
  images, in bright field, linear and circular polarized
  light. Routines were developed for the acquisition, registration,
  recognition and morphological analysis of hematite crystals. The
  automatic segmentation of hematite crystals was based on calculating
  the spectral distance, in order to control the region expansion from
  the seeds. The results regarding the identification of the obtained
  crystals were very promising. Size and shape attributes were
  obtained and used as a training set for supervised classifiers,
  leading to the recognition of granular, lamelar and lobular hematite
  classes. Global success rates close to 98\% were obtained concerning
  self-validation as well as crossed validation.}

%%%
%%% Epigrafe
%%%

%% \epigrafe{%
%%   no epigrafe}
%% \epigrafeautor{Wassily Kandinsky}
%% \epigrafelivro{Regards sur le passé}

%%%
%%% Misc.
%%%

\usecolour{true}

%%%
%%% Hyphenation
%%%

\hyphenation{PON-TI-FÍ-CIA}

%%%
%%% 
%%%

\begin{document}

  % -*- coding: utf-8; -*-

\chapter{Introdução}

O ferro é o metal mais usado pela sociedade devido à alta
disponibilidade, pelas propriedades físicas (dutilidade,
maleabilidade, resistência mecânica, etc.), por sua importância na
produção de aço e ferro fundido, assim como pelas suas muitas
aplicações.

O uso do ferro vem desde a antiguidade. Provavelmente a primeira vez
que o homem fez contato com o ferro metálico foi sob a forma de
meteoritos, daí a etimologia da palavra siderurgia, cujo radical
latino \textit{sider} significa estrela ou astro. No antigo Egito
foram descobertos ornamentos de ferro datados de cerca de 4000 A.C.,
também na pirâmide de Gizé foram achadas peças datadas de 2900 A.C. A
primeira indústria do ferro apareceu ao sul do Cáucaso, 1700 A.C.,
entre os Hititas. Também, na Assíria, foram encontradas ferramentas de
aço que datavam de 700 anos A.C. \cite{1}

Na atualidade, o uso crescente do aço e do ferro fundido na fabricação
de produtos de consumo evidencia a importância da indústria
metalúrgica para a economia nacional e global. Em contrapartida, a
qualidade do minério de ferro disponível vem diminuindo ao longo dos
anos. Portanto, as empresas de mineração tem se esforçado para
aumentar a produção e melhorar seus produtos, a fim de manter e
aperfeiçoar o seu desempenho no mercado.

O minério de ferro é um material policristalino que passou por vários
processos naturais complexos. Estes processos ocorreram durante tempos
geológicos, devido aos efeitos da pressão, às mudanças de temperatura,
à recristalização e à erosão, dando origem a diversas características
intrínsecas, e consequentemente, a um comportamento industrial
variável.\cite{2}

Assim, o minério de ferro é normalmente utilizado de duas formas:
minérios granulados e minérios aglomerados. Os granulados (entre $25$
mm e $6$ mm) são adicionados diretamente nos fornos de redução,
enquanto os aglomerados são os minérios finos que devido à sua
granulometria necessitam de aglomeração. Daí surgem as denominações
\textit{sinter-feed} (entre $6,35$ mm e $0,15$ mm) e
\textit{pellet-feed} (menos de $0,15$ mm) que identificam as frações
usadas nos processos de sinterização e pelotização,
respectivamente.\cite{3}

Devido ao grande interesse econômico, bem como o seu desempenho
durante o processo, a caracterização de cada uma destas frações
adquire grande importância. No entanto, não existe um método universal
para isto. De fato, este é um problema complexo, uma vez que são
diversos os atributos que caracterizam forma, textura, trama ou
porosidade, assim como as maneiras como são combinados em cada
caso.\cite{4}

Os minerais carreadores de ferro mais comuns (hematita, magnetita e
goethita) podem ser identificados visualmente no Microscópio de Luz
Refletida (MLR) através de suas refletâncias distintas.\cite{5} Assim,
a caracterização qualitativa de minérios de ferro é geralmente
realizada através de avaliação visual no MLR.

Na indústria mineral, a caracterização microestrutural (mineralógica e
textural) do minério de ferro e seus aglomerados é tradicionalmente
realizada por operadores humanos, pela observação de amostras ao MLR,
para identificar as fases presentes e estimar suas frações. Esse é um
procedimento rotineiro, realizado algumas vezes por dia e
consequentemente suscetível a falhas decorrentes da fadiga humana,
além de erros aleatórios diversos. Deste modo, tem havido um interesse
crescente no desenvolvimento de sistemas automáticos de análise
quantitativa que possam conferir maior reprodutibilidade,
confiabilidade e velocidade.

Por outro lado, sistemas automáticos de análise digital de imagens são
capazes de identificar hematita, magnetita e goethita pelas suas
tonalidades em imagens obtidas pelo MLR. Estes sistemas têm a vantagem
de serem mais velozes, práticos e reprodutíveis do que um operador
humano. Nos últimos anos, algumas metodologias foram desenvolvidos
para realizar a caracterização mineralógica de minérios de ferro
através de sistemas de análise de imagens.\cite{6,7,8,9}

A grande maioria dos minérios de ferro brasileiros é essencialmente
hematítica, geralmente envolvendo outros minerais como magnetita,
goethita e minerais de ganga, principalmente quartzo. No entanto,
estes minérios apresentam grande diversidade de microestruturas. A
hematita, por exemplo, pode ser classificada como lobular, lamelar,
granular, microcristalina ou martita.

O tamanho, a forma e a distribuição dos cristais de hematita podem
influenciar na redutibilidade e resistência mecânica dos
aglomerados. Por exemplo, hematitas granular e lamelar aumentam a
resistência mecânica dos aglomerados, mas reduzem sua porosidade e sua
redutibilidade. Já a hematita martítica age no sentido oposto,
aumentando a porosidade e redutibilidade dos aglomerados, mas
reduzindo sua resistência mecânica.\cite{2,10,11} Assim, a
determinação das características texturais da hematita certamente
contribui para um melhor conhecimento dos minérios de ferro, abrindo
novas possibilidades a fim de aprimorar seu processamento.\cite{12}

A hematita é um mineral fortemente anisotrópico. Ela apresenta
1pleocroísmo de reflexão (birrefletância), ou seja, sua refletância e,
consequentemente, o seu brilho na imagem mudam com diferentes
orientações dos cristais em relação ao plano de incidência da
luz.\cite{5} Essa variação de brilho é sutil, mas é perceptível ao
olho humano treinado no MLR.

Por sua vez, o uso combinado de um polarizador e um analisador no MLR
gera variações de brilho e cores devido à anisotropia.\cite{13} Esta
abordagem pode ser usada para obter imagens que apresentam um
contraste suficiente para diferençar os cristais.

Como já constatado, os minérios de ferro podem ter uma estrutura muito
complexa, com a associação de diferentes minerais e texturas. De tal
modo, não é muito difícil imaginar que criar um algoritmo de análise
de imagens capaz de identificar e caracterizar todas as formas de
hematita é um grande desafio. Sendo assim, este trabalho tem como
objetivo desenvolver uma metodologia de aquisição, processamento e
análise de imagens no MLR para:

\begin{enumerate}[label=(\roman{*})]
  \item Segmentar cristais de hematita compacta (granular, lamelar e
    lobular) no minério de ferro em amostras ricas neste mineral;
  \item Medir tamanho de cristais de hematita compacta e;  
  \item Medir os cristais de hematita compacta visando a sua
    classificação segundo sua morfologia.
\end{enumerate}

O grupo de pesquisa em Microscopia Digital (MD) do DEMa/PUC-Rio está
tentando desenvolver uma metodologia de classificação automática de
tipos de hematita através de duas abordagens diferentes, analítica e
sintética. A primeira é objeto de estudo deste trabalho, já a segunda
será desenvolvida por outro integrante do grupo como parte da sua
dissertação de mestrado.

O método sintético consiste em empregar parâmetros de textura com o
objetivo de identificar tipos texturais de hematita não compacta
(martita e microcristalina). Com este fim, as imagens são analisadas
em \textit{textels} (elementos de textura), dos quais são extraídos os
parâmetros de textura. Estes parâmetros de textura são empregados como
atributos no sistema de classificação.

Por sua vez, o método analítico combina diversas imagens de um mesmo
campo, obtidas com e sem polarização. A imagem sem polarização permite
separar a hematita das demais fases, a partir de seu brilho. As
imagens com polarização permitem encontrar as fronteiras entre os
cristais de hematita. Uma vez separados os cristais de hematita, estes
são medidos e classificados segundo sua morfologia.\cite{2,13}

O método analítico tem como característica fundamental o
reconhecimento individual de cada cristal de hematita, porém o método
é ineficiente na identificação dos cristais das fases não compactas. É
assim que o método sintético visa complementar esta deficiência,
criando, no conjunto, uma metodologia de classificação automática dos
cinco tipos de hematita.

A presente tese está organizada em cinco capítulos. O primeiro
capítulo consiste desta introdução.

O segundo capítulo (``Revisão Bibliográfica") traz uma retrospectiva
do minério de ferro, sua importância econômica, sua composição e sua
microestrutura. Ao mesmo tempo, expõe o conteúdo teórico das diversas
técnicas experimentais envolvidas no trabalho e descreve algumas
técnicas que vêm sendo estudadas.

O terceiro capítulo (``Materiais e Métodos") descreve o mecanismo de
preparação das amostras de minério de ferro. Por sua vez, apresenta as
etapas experimentais, assim como os equipamentos e técnicas usadas na
análise destas amostras. Neste capítulo, são também descritas as
técnicas de identificação, medição e classificação dos cristais de
hematita.

O quarto capítulo (``Resultados e Discussões") apresenta e discute os
resultados. Neste capítulo, são expostas as vantagens e desvantagens
de cada técnica experimental, assim como sua eficiência na
identificação de cristais de hematita.

Finalmente, o quinto capítulo (``Conclusões") apresenta as conclusões
e propostas para trabalhos futuros.

\textit{Cumpre comentar, nesta introdução, sobre uma característica um
  tanto peculiar da estrutura do presente trabalho. Para atingir os
  objetivos listados acima foram desenvolvidos diferentes métodos de
  microscopia digital, envolvendo a criação de rotinas sofisticadas de
  processamento e análise de imagens. Assim, parte dos métodos
  "utilizados" foram, na verdade, "desenvolvidos" no decorrer do
  trabalho. Por esta razão, os conteúdos dos capítulos de Materiais e
  Métodos e de Resultados muitas vezes se misturam e retroalimentam,
  gerando uma estrutura um tanto fora do padrão.}

  % -*- coding: utf-8; -*-

\chapter{Review}

This is the second chapter...

In this chapter, let's have a nice table:

%% -*- coding: utf-8; -*-

\begin{table} [!h]
  \caption{Principais minerais de ferro e suas classes.\cite{29}}\label{tab:2-4}
  ~\\[-1mm]
   \begin{tabularx}
     {\textwidth}
     { p{2.0cm}
       p{2.5cm}
       p{3.3cm}
       p{1.3cm}
       p{2.7cm}}

     \textbf{Classes}
     & \textbf{Minerais}
     & \textbf{\mrcel {Fórmula}{Química}}
     & \textbf{\mrcel{Teor}{de Fe}}
     & \textbf{\mrcel{~~Designação}{~~Comum}}
     \\\toprule

     ~ \\[-6mm]
     \multirow{5}{*}{Óxidos}& Magnetita
     & $Fe_{3}O_{4}$
     & ~72,4
     & \mrcel{~~Óxido ferroso}{~~férrico}
     \\%\midrule

     & Hematita
     & $Fe_{2}O_{3}$
     & ~69,9
     & ~~Óxido férrico \\[2mm]

     & Goethita
     & $FeO(OH)$
     & ~62,8
     & \multirow{2}{*}{\mrcel{Óxido-hidróxido}{de ferro}} \\[2mm]


     & Lepidocrocita
     & $FeO(OH)$
     & ~62,8 &
     \\\midrule

     Carbonato
     & Siderita
     & $FeCO_{3}$
     & ~48,2
     & \mrcel{~~~~Carbonato}{~~~~de Ferro}
     \\\midrule

     \multirow{2}{*}{Sulfetos}
     & Pirita
     & $FeS_{2}$
     & ~46,5
     & \multirow{2}{*}{~} \\[2mm]


     & Pirrotita
     & $FeS$
     & ~63,6
     & ~
     \\\midrule

     \multirow{10}{*}{Silicatos}
     & Fayalita
     & $Fe^{2+}_{2}(SiO_{4})$
     & ~54,8
     & \mrcel{~~~~Grupo da}{~~~~Olivina} \\[4mm]

     & Laihunite
     & $Fe^{2+}Fe^{3+}_{2}(SiO_{4})_{2}$
     & ~47,6
     & \mrcel{~~~~Grupo da}{~~~~Olivina} \\[4mm]

     & Greenalita
     & \mrcell{$2Fe^{2+}_{2}6Fe^{3+}Si_{2}$}{$4O_{5}(OH)_{3,3}$}
     & ~44,1
     & \mrcel{~~~~Grupo da}{~~~~Serpentina} \\[4mm]

     & Grunerita
     & \mrcell{$Fe^{2+}_{7}(Si_{8}O_{22})$}{$(OH)_{2}$}
     & ~39,0
     & \mrcel{~~~~Grupo dos}{~~~~Anfibólios} \\[4mm]

     & Fé-antofilita
     & \mrcell{$Fe^{2+}_{7}(Si_{8}O_{22})$}{$(OH)_{2}$}
     & ~39,0
     & \mrcel{~~~~Grupo dos}{~~~~Anfibólios}
     \\\midrule
   \end{tabularx}
\end{table}

% -*- coding: utf-8; -*-

\newcommand{\coltworowone}{%
\begin{tabular}{ l @{\extracolsep{2mm}}X }
  \mybulletOB
    & Cristais ~~ muito	\\[-1.2mm]
  ~ & pequenos, 	\\[-.4mm]
  ~ & $<0.01$ mm. \\
  \mybulletOB & Textura porosa. \\
  \mybulletOB
    & Contatos pouco	\\[-1.2mm]
  ~ & desenvolvidos.
\end{tabular}}

\newcommand{\coltworowtwo}{%
\begin{tabular}{ l @{\extracolsep{2mm}}X }
  \mybulletOB
    & Cristais euédricos \\[-1.2mm]
  ~ & isolados ~ ou ~ em \\[-1.2mm]
  ~ & agregados. \\
  \mybulletOB
    & Cristais compac- \\[-1.2mm]
  ~ & tos.
\end{tabular}}

\newcommand{\coltworowthree}{%
\begin{tabular}{ l @{\extracolsep{2mm}}X }
  \mybulletOB
    & Hematita ~~ com \\[-1.2mm]
  ~ & hábito de magne- \\[-1.2mm]
  ~ & tita. \\
  \mybulletOB
    & Oxidação segundo \\[-1.2mm]
  ~ & os planos ~ crista- \\[-1.2mm]
  ~ & lográficos da mag- \\[-1.2mm]
  ~ & netita. \\
  \mybulletOB
    & Geralmente ~ po- \\[-1.6mm]
  ~ & rosa.
\end{tabular}}

\newcommand{\coltworowfour}{%
\begin{tabular}{ l @{\extracolsep{2mm}}X }
  \mybulletOB
    & Formatos irregu- \\[-1.2mm]
  ~ & lares ~~ inequidi- \\[-1.2mm]
  ~ & mensionais. \\
  \mybulletOB
    & Contatos irregula-\\[-1.2mm]
  ~ & res, ~ geralmente \\[-1.2mm]
  ~ & imbricados.
 \end{tabular}}

\newcommand{\coltworowfive}{%
\begin{tabular}{ l @{\extracolsep{2mm}}X }
  \mybulletOB
    & Formatos regula- \\[-1.2mm]
  ~ & res ~~ equidimen- \\[-1.2mm]
  ~ & sionais. \\
  \mybulletOB
    & Contatos ~~ retilí- \\[-1.2mm]
  ~ & neos ~ e ~ junções \\[-1.2mm]
  ~ & tríplices. \\
  \mybulletOB
    & Cristais compac-\\[-1.6mm]
  ~ & tos.
 \end{tabular}}

\newcommand{\coltworowsix}{%
\begin{tabular}{ l @{\extracolsep{2mm}}X }
  \mybulletOB
    & Cristais inequidi- \\[-1.2mm]
  ~ & mensionais, hábi- \\[-1.2mm]
  ~ & to tabular. \\
  \mybulletOB
    & Contato retilíneo. \\
  \mybulletOB
    & Cristais compac- \\[-1.6mm]
  ~ & tos.
 \end{tabular}}

\newcommand{\coltworowseven}{%
\begin{tabular}{ l @{\extracolsep{2mm}}X }
  \mybulletOB
    & Material cripto- \\[-1.2mm]
  ~ & cristalino.  \\
  \mybulletOB
    & Estrutura colofor- \\[-1.2mm]
  ~ & me, hábito botri- \\[-1.2mm]
  ~ & oidal.  \\
  \mybulletOB
    & Textura porosa.
 \end{tabular}}

\begin{table} [!p]
    \caption{Main morphologies of hematite.\cite{14}}\label{tab:2-5}
    ~\\[-2mm]
  \begin{tabularx}{\textwidth}{@{\extracolsep{0pt}}C @{\extracolsep{0pt}}C C C}

    \textbf{Tipo}
    & \textbf{Características}
    & \textbf{\mrcel{Forma}{Textura}}
    & \textbf{\mrcel{Ilustração}{Esquemática}}
    \\\toprule

    ~ \\[-6mm]
    \mrcel{Hematita}{Microcristalina}
    & \coltworowone
    & \mytbcimg{2.3cm}{2.9cm}{images/Microcristalina}
    & \mytbcimg{2.6cm}{2.5cm}{images/MicrocristalinaEsq}
    \\\midrule

    ~\\[-6mm]
    Magnetita
    & \coltworowtwo
    & \mytbcimg{2.3cm}{2.9cm}{images/Magnetita}
    & \mytbcimg{2.6cm}{2.9cm}{images/MagnetitaEsq}
    \\\midrule

    ~\\[-5mm]
    Martita
    & \coltworowthree
    & \mytbcimg{2.3cm}{2.9cm}{images/Martita}
    & \mytbcimg{2.6cm}{2.9cm}{images/MartitaEsq}
    \\\midrule

    ~\\[-5mm]
    \mrcel{Hematita}{Lobular}
    & \coltworowfour
    & \mytbcimg{2.3cm}{2.9cm}{images/Lobular}
    & \mytbcimg{2.6cm}{2.9cm}{images/LobularEsq}
    \\\midrule

    ~\\[-5mm]
    \mrcel{Hematita}{Granular}
    & \coltworowfive
    & \mytbcimg{2.3cm}{2.9cm}{images/Granular}
    & \mytbcimg{2.6cm}{2.9cm}{images/GranularEsq}
    \\\midrule

    ~\\[-5mm]
    \mrcel{Hematita}{Lamelar}
    & \coltworowsix
    & \mytbcimg{2.3cm}{2.9cm}{images/Lamelar}
    & \mytbcimg{2.6cm}{2.9cm}{images/LamelarEsq}
    \\\midrule

    ~\\[-5mm]
    \mrcelthree{Hidróxidos de}{ Fe (Goethita-}{Limonita)}
    & \coltworowseven
    & \mytbcimg{2.3cm}{2.9cm}{images/Goethita}
    & \mytbcimg{2.6cm}{2.9cm}{images/GoethitaEsq}
    \\\midrule

  \end{tabularx}
\end{table}


\section{Hematite}

A hematita é o mineral de ferro mais importante devido a sua alta
ocorrência em vários tipos de rochas e suas origens diversas.\cite{30}
A composição química deste mineral é Fe$_{2}$O$_{3}$, com uma fração
mássica em ferro de 69,9\% e uma fração mássica em oxigênio de
30,1\%.\cite{31}

...


\subsection{Martite}

A hematita é o mineral de ferro mais importante devido a sua alta
ocorrência em vários tipos de rochas e suas origens diversas.\cite{30}
A composição química deste mineral é Fe$_{2}$O$_{3}$, com uma fração
mássica em ferro de 69,9\% e uma fração mássica em oxigênio de
30,1\%.\cite{31}

...


\subsubsection{Globular}

A hematita é o mineral de ferro mais importante devido a sua alta
ocorrência em vários tipos de rochas e suas origens diversas.\cite{30}
A composição química deste mineral é Fe$_{2}$O$_{3}$, com uma fração
mássica em ferro de 69,9\% e uma fração mássica em oxigênio de
30,1\%.\cite{31}

...

\subsubsection{Escaping percent in a title:  100\%}

  %% \input{chapter-3}
  %% \input{chapter-4}
  % -*- coding: utf-8; -*-

\chapter{Conclusão e Trabalhos Futuros}

Um sistema de microscopia digital com reconhecimento e classificação
automática dos cristais de hematita em minérios de ferro foi
desenvolvido.

O método utiliza operações tradicionais de processamento digital de
imagens e propõe uma segmentação automática de cristais baseada no
cálculo da distância espectral, a fim de controlar o crescimento das
regiões partindo das sementes. É importante salientar que o método de
crescimento de regiões proposto é muito robusto e capaz de lidar com o
grande número de sementes derivado do método de \textit{watersheds}.

O método proposto também envolve técnicas de microscopia digital,
permitindo a segmentação dos cristais tanto das imagens com polarização
linear quanto das imagens com polarização circular.

Os resultados obtidos para ambos os tipos de imagens são muito
promissores. A grande maioria dos cristais foi corretamente
identificada. Mesmo cristais adjacentes com cores similares foram
corretamente segmentados. O método é completamente automático com um
único parâmetro t para ser ajustado, o qual controla a sensibilidade da
distancia Euclidiana do pixel no espaço RGB.

Medidas morfológicas dos cristais foram obtidas com sucesso. Estas
medidas foram analisadas e estudadas estatisticamente.

Os resultados obtidos para a classificação das classes pertencentes a
ambos os tipos de imagens foram muito bons. Taxas de acerto globais
próximas a 98\% foram obtidas, tanto para a autovalidação, quanto para a
validação cruzada.

No entanto, algumas limitações devem ser apontadas. Dada a sua
característica de segmentação, o método também é sensível a problemas de
preparação de amostras, tais como problemas de relevo ou arranhões. Isto
chama a atenção para uma cuidadosa preparação de amostras, sendo que,
esta operação é de vital importância para a realização de qualquer tipo
de análise de imagens automática.

Outro método mais simples de segmentação de cristais também foi
desenvolvido. Este método é mais veloz em sua etapa automática, mas
exige uma trabalhosa etapa de ajuste manual, o que o torna menos
prático.

É fundamental também comentar que o sistema proposto é limitado a
classificar e medir cristais das fases compactas da hematita - granular,
lamelar e lobular - e não se propõe a identificar e quantificar hematita
microcristalina e martita. Como comentado na introdução desta tese, um
outro enfoque, denominado método sintético, busca incluir estas outras
classes no sistema através de uma análise de textura (no sentido da área
de PADI e não no sentido de textura microestrutural). Resultados
preliminares indicam que este enfoque é promissor, sendo capaz de
discriminar as 5 classes citadas com boa taxa de acerto. No entanto,
justamente por ter um enfoque sintético, o método não é capaz de
discriminar e medir cristais individuais, o que é uma limitação
importante.

Assim, como uma proposta para trabalho futuro, pode-se buscar combinar
os dois enfoques. O enfoque sintético seria capaz de discriminar as
fases compactas e porosas e, em seguida, as fases compactas seriam
submetidas ao método analítico proposto na presente tese, para
discriminação e medida dos parâmetros morfológicos dos cristais
individuais.

  \arial
  \bibliography{julio}
  \normalfont
  % -*- coding: utf-8; -*-

\appendix
\chapter{Artigo da Tese Publicado em Periódico}

O método de segmentação Crescimento de Regiões Modificado mostrado neste
trabalho rendeu uma publicação no periódico \textit{Minerals
  Engineering}. Esta publicação serve como requisito parcial para
obtenção do título de Doutor pelo Programa de Pós-graduação em
Engenharia de Materiais e de Processos Químicos e Metalúrgicos da
PUC-Rio. É por isto que esta publicação será anexada à continuação desta
página.



\end{document}
