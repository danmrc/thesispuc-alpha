% -*- coding: utf-8; -*-

\chapter{Conclusão e Trabalhos Futuros}

Um sistema de microscopia digital com reconhecimento e classificação
automática dos cristais de hematita em minérios de ferro foi
desenvolvido.

O método utiliza operações tradicionais de processamento digital de
imagens e propõe uma segmentação automática de cristais baseada no
cálculo da distância espectral, a fim de controlar o crescimento das
regiões partindo das sementes. É importante salientar que o método de
crescimento de regiões proposto é muito robusto e capaz de lidar com o
grande número de sementes derivado do método de \textit{watersheds}.

O método proposto também envolve técnicas de microscopia digital,
permitindo a segmentação dos cristais tanto das imagens com polarização
linear quanto das imagens com polarização circular.

Os resultados obtidos para ambos os tipos de imagens são muito
promissores. A grande maioria dos cristais foi corretamente
identificada. Mesmo cristais adjacentes com cores similares foram
corretamente segmentados. O método é completamente automático com um
único parâmetro t para ser ajustado, o qual controla a sensibilidade da
distancia Euclidiana do pixel no espaço RGB.

Medidas morfológicas dos cristais foram obtidas com sucesso. Estas
medidas foram analisadas e estudadas estatisticamente.

Os resultados obtidos para a classificação das classes pertencentes a
ambos os tipos de imagens foram muito bons. Taxas de acerto globais
próximas a 98\% foram obtidas, tanto para a autovalidação, quanto para a
validação cruzada.

No entanto, algumas limitações devem ser apontadas. Dada a sua
característica de segmentação, o método também é sensível a problemas de
preparação de amostras, tais como problemas de relevo ou arranhões. Isto
chama a atenção para uma cuidadosa preparação de amostras, sendo que,
esta operação é de vital importância para a realização de qualquer tipo
de análise de imagens automática.

Outro método mais simples de segmentação de cristais também foi
desenvolvido. Este método é mais veloz em sua etapa automática, mas
exige uma trabalhosa etapa de ajuste manual, o que o torna menos
prático.

É fundamental também comentar que o sistema proposto é limitado a
classificar e medir cristais das fases compactas da hematita - granular,
lamelar e lobular - e não se propõe a identificar e quantificar hematita
microcristalina e martita. Como comentado na introdução desta tese, um
outro enfoque, denominado método sintético, busca incluir estas outras
classes no sistema através de uma análise de textura (no sentido da área
de PADI e não no sentido de textura microestrutural). Resultados
preliminares indicam que este enfoque é promissor, sendo capaz de
discriminar as 5 classes citadas com boa taxa de acerto. No entanto,
justamente por ter um enfoque sintético, o método não é capaz de
discriminar e medir cristais individuais, o que é uma limitação
importante.

Assim, como uma proposta para trabalho futuro, pode-se buscar combinar
os dois enfoques. O enfoque sintético seria capaz de discriminar as
fases compactas e porosas e, em seguida, as fases compactas seriam
submetidas ao método analítico proposto na presente tese, para
discriminação e medida dos parâmetros morfológicos dos cristais
individuais.
